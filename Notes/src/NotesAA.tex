\documentclass[11pt,a4paper]{article}

\usepackage[margin=1in, paperwidth=8.3in, paperheight=11.7in]{geometry}
\usepackage{amsfonts}
\usepackage{amsmath}
\usepackage{amssymb}
\usepackage{dsfont}
\usepackage{enumerate}
\usepackage{enumitem}
\usepackage{fancyhdr}
\usepackage{graphicx}
\usepackage{tikz}
\usepackage{changepage} 

\begin{document}

\pagestyle{fancy}
\setlength\parindent{0pt}
\allowdisplaybreaks

\renewcommand{\headrulewidth}{0pt}
\setlist[enumerate,1]{label={\roman*)}}

% Cover page title
\title{Advanced Algorithms - Notes}
\author{Dom Hutchinson}
\date{\today}
\maketitle

% Header
\fancyhead[L]{Dom Hutchinson}
\fancyhead[C]{Advanced Algorithms - Notes}
\fancyhead[R]{\today}

% Counters
\newcounter{definition}[subsection]
\newcounter{example}[subsection]
\newcounter{notation}[subsection]
\newcounter{proposition}[subsection]
\newcounter{proof}[subsection]
\newcounter{remark}[subsection]
\newcounter{theorem}[subsection]

% commands
\newcommand{\dotprod}[0]{\boldsymbol{\cdot}}
\newcommand{\cosech}[0]{\mathrm{cosech}\ }
\newcommand{\cosec}[0]{\mathrm{cosec}\ }
\newcommand{\sech}[0]{\mathrm{sech}\ }
\newcommand{\prob}[0]{\mathbb{P}}
\newcommand{\nats}[0]{\mathbb{N}}
\newcommand{\cov}[0]{\mathrm{Cov}}
\newcommand{\var}[0]{\mathrm{Var}}
\newcommand{\expect}[0]{\mathbb{E}}
\newcommand{\reals}[0]{\mathbb{R}}
\newcommand{\integers}[0]{\mathbb{Z}}
\newcommand{\indicator}[0]{\mathds{1}}
\newcommand{\nb}[0]{\textit{N.B.} }
\newcommand{\ie}[0]{\textit{i.e.} }
\newcommand{\eg}[0]{\textit{e.g.} }
\newcommand{\X}[0]{\textbf{X}}
\newcommand{\x}[0]{\textbf{x}}
\newcommand{\iid}[0]{\overset{\text{iid}}{\sim}}
\newcommand{\proved}[0]{$\hfill\square$\\}

\newcommand{\definition}[1]{\stepcounter{definition} \textbf{Definition \arabic{subsection}.\arabic{definition}\ - }\textit{#1}\\}
\newcommand{\definitionn}[1]{\stepcounter{definition} \textbf{Definition \arabic{subsection}.\arabic{definition}\ - }\textit{#1}}
\newcommand{\proof}[1]{\stepcounter{proof} \textbf{Proof \arabic{subsection}.\arabic{proof}\ - }\textit{#1}\\}
\newcommand{\prooff}[1]{\stepcounter{proof} \textbf{Proof \arabic{subsection}.\arabic{proof}\ - }\textit{#1}}
\newcommand{\example}[1]{\stepcounter{example} \textbf{Example \arabic{subsection}.\arabic{example}\ - }\textit{#1}\\}
\newcommand{\examplee}[1]{\stepcounter{example} \textbf{Example \arabic{subsection}.\arabic{example}\ - }\textit{#1}}
\newcommand{\notation}[1]{\stepcounter{notation} \textbf{Notation \arabic{subsection}.\arabic{notation}\ - }\textit{#1}\\}
\newcommand{\notationn}[1]{\stepcounter{notation} \textbf{Notation \arabic{subsection}.\arabic{notation}\ - }\textit{#1}}
\newcommand{\proposition}[1]{\stepcounter{proposition} \textbf{Proposition \arabic{subsection}.\arabic{proposition}\ - }\textit{#1}\\}
\newcommand{\propositionn}[1]{\stepcounter{proposition} \textbf{Proposition \arabic{subsection}.\arabic{proposition}\ - }\textit{#1}}
\newcommand{\remark}[1]{\stepcounter{remark} \textbf{Remark \arabic{subsection}.\arabic{remark}\ - }\textit{#1}\\}
\newcommand{\remarkk}[1]{\stepcounter{remark} \textbf{Remark \arabic{subsection}.\arabic{remark}\ - }\textit{#1}}
\newcommand{\theorem}[1]{\stepcounter{theorem} \textbf{Theorem \arabic{subsection}.\arabic{theorem}\ - }\textit{#1}\\}
\newcommand{\theoremm}[1]{\stepcounter{theorem} \textbf{Theorem \arabic{subsection}.\arabic{theorem}\ - }\textit{#1}}

\tableofcontents

% Start of content
\newpage

\setcounter{section}{-1}

\section{Reference}

\subsection{Probability}

\definition{Sample Space}
A \textit{Sample Space} is the set of possible outcomes of a scenario. A \textit{Sample Space} is not necessarily finite.\\
\eg Rolling a dice $S:=\{1,2,3,4,5,6\}$.\\

\definition{Probability Measure, $\prob$}
\textit{Probability Measure}, $\prob$, is a function from the sample space to $[0,1]$ which fulfils ${\displaystyle\sum_{x\in S}\prob(x)=1}$.\\
$$\prob:S\to[0,1]$$
A \textit{Probabiltiy Measure} must fulfil the criteria that for disjoint events $\{A_1,\dots,A_n\}$
$$\prob\left(\bigcup_iA_i\right)=\sum_i\prob(A_i)$$

\definition{Event}
An \textit{Event} is a subset of the \textit{Sample Space}.\\
The probability of an \textit{Event}, $A$, happening is
$$\prob(A)=\sum_{x\in A}\prob(x)$$

\definition{Sigma Field, $\mathcal{F}$}
A \textit{Sigma Field} is the set of possible events in a given scenario.\\
A \textit{Sigma Field} must fulfil the following criteria
\begin{enumerate}
	\item $S\in\mathcal{F}$.
	\item $\forall\ A\in\mathcal{F}\implies A^c\in\mathcal{F}$.
	\item $\forall\ A_1,\dots,A_n\in\mathcal{F}\implies\bigcup_{i}A_i\in\mathcal{F}$.
\end{enumerate}

\definition{Random Variable}
A \textit{Random Variable} is a function from the sample space, $S$, to the real numbers, $\reals$.\\
$$X:S\to\reals$$
The probability of a \textit{Random Variable}, $X$, taking a specific value $x$ is found by
$$\prob(X=x)=\sum_{\{a\in S:X(a)=x\}}\prob(a)$$

\definition{Indicator Random Variable}
An \textit{Indicator Random Variable} is a \textit{Random Variable} which only ever takes $0$ or $1$ and is used to indicate whether a particular event has happened (1), or not (0).
$$\expect(I)=\prob(I=1)$$

\definition{Expected Value, $\expect$}
The \textit{Expected Value} of a \textit{Random Variable} is the mean value of said \textit{Random Variable}
$$\expect(X):=\sum_x x\prob(X=x)$$

\theorem{Linearity of Expected Value}
Let $X_1,\dots,X_n$ be random variables. Then
$$\expect\left(\sum_{i=1}^nX_i\right)=\sum_{i=1}^n\expect(X_i)$$

\theorem{Markov's Inequality}
Let $X$ be a non-negative random variable. Then
$$\prob(X\geq a)\leq\frac1a\expect(X)\quad\forall\ a>0$$

\end{document}
